La prima osservazione da fare riguardo il worm Code Red è la velocità con cui è riuscito a diffondersi, almeno nella sua seconda versione. I confini fisici e geografici hanno perso di significato davanti ad un attacco di questa portata, e in meno di 14 ore più di 359,104 macchine erano state compromesse.\\
Un secondo aspetto da evidenziare è che le macchine gestite da utenti comuni in ambito domestico, o da piccole imprese, sono parte integrante della solidità di Internet globale. Queste sono spesso gli host più vulnerabili e possono essere attaccati con facilità da un attaccante, diventando una porta di accesso verso tutti gli altri host e quindi mettendo a rischio l’intera rete.\\
Il caso studiato del worm Code Red è un ottimo esempio di questa situazione, e mette in risalto un ulteriore aspetto che è purtroppo molto diffuso ed è alla base di altri attacchi informatici: la negligenza degli utenti. Gli utenti medi della rete tendono a dare poca importanza a molti aspetti della sicurezza informatica, anche di semplice tipo, come potrebbe essere la buona gestione delle proprie password, esponendo se stessi e quindi tutto il resto della rete a possibili attacchi. Nel caso di Code Red abbiamo visto, tra le contromisure attuabili, il protocollo SSL, che è un esempio perfetto di quanto appena detto. Esso avrebbe evitato la diffusione di Code Red, e quindi tutte le conseguenze economiche dell’attacco (che ricordiamo essere state stimate intorno ai 2.6 miliardi di dollari) se fosse stato adottato da un’elevata quantità di utenti. Eppure nonostante il suo costo praticamente irrisorio, al momento dell'attacco meno della metà dell’1% degli utenti disponeva di tale protocollo attivo, e questo è uno dei motivi per cui Code Red non ha trovato freni nella sua diffusione planetaria.\\
Come visto le possibili contromisure non sono poche e risultano essere efficaci per arginare o bloccare Code Red sul nascere della sua diffusione. Le ACL attualmente sarebbero una soluzione ottima dato che il pattern di Code Red è ormai noto, ma nel 2001 non sarebbero state troppo di aiuto. La soluzione più efficace per evitare la diffusione e bloccarla sul nascere sarebbe di sicuro la più costosa, cioè l’utilizzo di un sistema IDS, anche se visto il suo costo di configurazione e di gestione lo renderebbe applicabile a zone limitate della rete.\\
Vista l’impossibilità di riconoscere un pattern sconosciuto e i costi molto elevati, la soluzione migliore per poter mitigare la diffusione del worm e un attacco DDoS, sia da un punto di vista realizzativo che economico, potrebbe essere rappresentata dall’uso di firewall, sia a livello di rete che applicativo, per poter filtrare il traffico ed evitare il carico diretto dell’attacco sul dispositivo target.\\
Per evitare la diffusione del worm invece, il protocollo SSL resta una delle soluzioni più efficaci ed economiche. 
Infine non andrebbe dimenticato il contenimento automatizzato, che nonostante sia applicabile soltanto a worm che eseguono scansioni randomiche, rimane comunque una soluzione flessibile, poco invasiva e soprattutto che non ha bisogno di conoscere le signature di eventuali minacce come Code Red.\\
