In generale non è facile trovare il responsabile di un malware di tipo worm, dato che questo non legandosi a file nei dispositivi che infetta, non lascia una “scia” dietro di sé da sfruttare per risalire al primo dispositivo infetto, il “paziente zero”, ed eventualmente al suo creatore.\\
Anche per il caso di Code Red è stato possibile solo fare delle ipotesi, e la prima è stata fatta proprio dalla eEye.
La eEye ha sostenuto che il worm è stato creato a Makati City, Filippine, analogamente al worm “ILOVEYOU”.\\ 
Keith Rhodes, capo tecnico del General Accounting Office ha invece rilasciato un report in cui ha ipotizzato che Code Red è stato sviluppato presso l’Università Foshan in Guandong, Cina.\\
La Cina è stata uno dei primi paesi sospettati per lo sviluppo di Code Red, a causa del messaggio che veniva visualizzato dai dispositivi infettati all’accesso sulla rete (n.d.r. “Hacked by Chinese. Welcome to http://www.worm.com.”). Inoltre il lancio del virus è avvenuto poco dopo un incidente aereo che ha coinvolto un aereo spia militare americano e un jet da combattimento cinese.
Il personale dell’Università Foshan dichiarò però che, al lancio del worm, i laboratori si trovavano in ristrutturazione e l’Università era nel periodo di vacanza, facendo quindi perdere di credibilità questa possibile soluzione.\\
A favore di questa tesi ci furono delle rilevazioni eseguite da Dshield.org, le quali dimostrarono che il virus colpì prima gli Stati Uniti e altri Paesi prima di arrivare in Cina. Dunque l’origine di Code Red andrebbe cercata altrove.\\
Den Eichman, ingegnere capo della sicurezza per il CAS, sottolineò che l’infezione poteva essere partita anche da remoto, rendendo di fatto impossibile definire con sicurezza da dove fosse partito il worm.\\
Johannes Ullrich, operatore di Dshield.org ipotizzò invece che il worm fu lanciato da uno dei partecipanti al DefCon, convention annuale di hacker che si svolse a Las Vegas il 13 luglio.\\
