A luglio del 2001 più di 359.000 computer sono stati infettati, in meno di 14 ore, da un worm chiamato “Code Red”.\\
Per infettare un dispositivo Code Red sfruttava una vulnerabilità dei server IIS di Microsoft, e la vulnerabilità riguardava l’overflow di un buffer usato nella ricezione delle richieste per il server. I dispositivi infettati venivano poi utilizzati per effettuare un attacco DDoS ai danni della Casa Bianca.\\
E’ stato scoperto il 13 luglio 2001 da Marc Maiffret e Myan Parmah, dipendenti di eEye Digital Security ed è diventato famoso come il malware più costoso del 2001.\\
Lo scopo di questo documento è di trattare l'incidente in maniera approfondita, partendo dalla descrizione dettagliata del comportamento del worm e dalla vulnerabilità sfruttata, per poi analizzare l'impatto che ha avuto sulla rete globale, quali sono stati i fattori che hanno permesso di provocare tanti danni e quali sono state le contromisure intraprese nel tentativo di porre fine alla sua diffusione. Infine abbiamo provato ad immaginare cosa sarebbe successo se il worm avesse agito in maniera indisturbata, e quali alternative si sarebbero potute adottare per limitare i danni in maniera più efficace.\\
